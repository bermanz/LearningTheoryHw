\subsection*{a}
\begin{proof}    
    Let $h(\pmb{w}) = \frac{1}{m}\sum_{i=1}^m \ell\left(\pmb{w}, (x_i, y_i)\right)$. Since $\ell\left(\pmb{w}, (x, y)\right)$ is convex, so is $h(\pmb{w})$ (as an affine combination of convex functions), and hence:
    \begin{equation*}
        h(\pmb{w}) - h(\pmb{v}) \geq \nabla h(\pmb{v})^T \cdot (\pmb{w}-\pmb{v})
    \end{equation*}
    
    In addition, we note that:
    
    \begin{equation*}
        \begin{split}            
            ||\pmb{w}||^2 - ||\pmb{v}||^2 &= 2<\pmb{w}, \pmb{v}> - 2||\pmb{v}||^2 + ||\pmb{w} - \pmb{v}||^2 \\
            &= 2\left(\pmb{v}^T \cdot (\pmb{w} - \pmb{v}) \right) + ||\pmb{w} - \pmb{v}||^2 \\
            &= \nabla(||\pmb{v}||^2)^T \cdot (\pmb{w} - \pmb{v}) + ||\pmb{w} - \pmb{v}||^2
        \end{split}
    \end{equation*}
    
    Plugging all in, we have:
    \begin{equation*}
        \begin{split}
            F_\lambda(\pmb{w}) - F_\lambda(\pmb{v}) &= \frac{\lambda}{2}(||\pmb{w}||^2 - ||\pmb{v}||^2) + h(\pmb{w}) - h(\pmb{v}) \\ 
            &= \frac{\lambda}{2}\left(\nabla(||\pmb{v}||^2)^T \cdot (\pmb{w} - \pmb{v}) + ||\pmb{w} - \pmb{v}||^2\right) + h(\pmb{w}) - h(\pmb{v}) \\
            &\geq \frac{\lambda}{2}\left(\nabla(||\pmb{v}||^2)^T \cdot (\pmb{w} - \pmb{v}) + ||\pmb{w} - \pmb{v}||^2\right) + \nabla h(\pmb{v})^T \cdot (\pmb{w}-\pmb{v}) \\
            &= \left(\nabla(\frac{\lambda}{2}||\pmb{v}||^2 + h(\pmb{v}))^T\right) \cdot (\pmb{w} - \pmb{v}) + \frac{\lambda}{2}||\pmb{w} - \pmb{v}||^2 \\
            &= \nabla F_\lambda(\pmb{v})^T \cdot(\pmb{w} - \pmb{v}) + \frac{\lambda}{2}||\pmb{w} - \pmb{v}||^2
        \end{split}
    \end{equation*}
    
    So we conclude that:
    \begin{equation*}
        F_\lambda(\pmb{w}) \geq F_\lambda(\pmb{v}) + \nabla F_\lambda(\pmb{v})^T \cdot(\pmb{w} - \pmb{v}) + \frac{\lambda}{2}||\pmb{w} - \pmb{v}||^2
    \end{equation*}
    Hence $F_\lambda(\pmb{w})$ is $\lambda$ strongly convex
\end{proof}

\subsection*{b}
First, let's calculate the gradient of $\ell\left(\pmb{w}, (x_i, y_i)\right)$:
\begin{equation*}
    \nabla \ell\left(\pmb{w}, (x_i, y_i)\right) = -y \frac{e^{-y \pmb{w} \pmb{x}}}{1+e^{-y \pmb{w} \pmb{x}}} \pmb{x}
    = \frac{-y}{1+e^{y \pmb{w} \pmb{x}}} \pmb{x}
\end{equation*}
Where the last transition is obtained by multiplying and dividing by $e^{y \pmb{w} \pmb{x}}$, which is non-zero for any possible combination of $y, \pmb{w},\pmb{x}$. Taking the standard norm of the gradient:
\begin{equation*}
    ||\nabla \ell\left(\pmb{w}, (x_i, y_i)\right)|| = ||\frac{-y}{1+e^{y \pmb{w} \pmb{x}}} \pmb{x}|| = |\frac{-y}{1+e^{y \pmb{w} \pmb{x}}}| ||\pmb{x}|| = |\frac{1}{1+e^{y \pmb{w} \pmb{x}}}| ||\pmb{x}|| \leq 1 \cdot B = B
\end{equation*}
Where the second-last transition is since $y \in \{-1, 1\}$ and the last transition since $0<|\frac{1}{1+e^{y \pmb{w} \pmb{x}}}| \leq 1$ and since it's given that $||\pmb{x}|| \leq B$. Hence, the parameter Liphschitzness of $\ell\left(\pmb{w}, (x_i, y_i)\right)$ is $B$.


For smoothness, we will use a known theorem by which a function is $\beta$ smooth if it is twice-continuously differentiable and it's hessian matrix satisfies:
\begin{equation*}
    \nabla^2_{\boldsymbol{w}} f(\boldsymbol{w}) \preceq \beta I
\end{equation*}
Let's calculate the hessian of $\ell(\boldsymbol{w}, (\boldsymbol{x}, y))$:

\begin{equation*}
    \nabla \ell(\boldsymbol{w}, (\boldsymbol{x}, y)) = \frac{-yexp(-y\boldsymbol{w}\cdot \boldsymbol{x})}{1+exp(-y\boldsymbol{w}\cdot \boldsymbol{x})} \boldsymbol{x}
    = \frac{-y}{1+exp(y\boldsymbol{w}\cdot \boldsymbol{x})} \boldsymbol{x}
\end{equation*}
Where the last transition is obtained by multiplying and dividing by $exp(y\boldsymbol{w}\cdot \boldsymbol{x})$.
\begin{equation*}
    \begin{split}        
        \nabla^2 \ell(\boldsymbol{w}, (\boldsymbol{x}, y)) &= \frac{y^2 exp(y\boldsymbol{w}\cdot \boldsymbol{x})}{(1+exp(y\boldsymbol{w}\cdot \boldsymbol{x}))^2} \boldsymbol{x} \cdot \boldsymbol{x}^T \\        
        &= \frac{exp(y\boldsymbol{w}\cdot \boldsymbol{x})}{(1+exp(y\boldsymbol{w}\cdot \boldsymbol{x}))^2} \boldsymbol{x} \cdot \boldsymbol{x}^T 
        \preceq \frac{exp(y\boldsymbol{w}\cdot \boldsymbol{x})}{(1+exp(y\boldsymbol{w}\cdot \boldsymbol{x}))^2} \cdot B^2 I
    \end{split}
\end{equation*}
Where the second last transition is correct since $y^2 = (\pm 1)^2 = 1$, and the last transition since the matrix $\boldsymbol{x} \cdot \boldsymbol{x}^T$ has a single non-zero eigen value which equalls $||x||^2=B^2$.


Now, let's annotate $exp(y\boldsymbol{w}\cdot \boldsymbol{x}) \triangleq m$, then:
\begin{equation*}
\nabla^2 \ell(m) \preceq B^2 \frac{m}{((1+m)^2)} I \triangleq B^2 f(m) I
\end{equation*}
In an attempt to upper-bound the term, let's see if the function $f(m)$ has a global maximum:

\begin{equation*}
    \frac{df(m)}{dm} = \frac{(1+m)^2 - 2m(1+m)}{(1+m)^4} 
    = \frac{m^2 + 2m + 1 - 2m -2m^2}{(1+m)^4} = \frac{1- m^2}{(1+m)^4} = \frac{1-m}{(1+m)^3}
\end{equation*}

Thus $m=1$ is the sole-suspected global extremum of the function. plugging in values around it:
\begin{equation*}
    \begin{split}
        f(1) &= \frac{1}{((1+1)^2)} = \frac{1}{4} \\
        f(0) &= \frac{0}{((1+0)^2)} = 0 < f(1) \\
        f(2) &= \frac{2}{((1+2)^2)} = \frac{2}{9} < f(1)
    \end{split}
\end{equation*}
Hence $m=1$ is the global maximum of $f(m)$, and so:

\begin{equation*}
    \nabla^2 \ell(m) \preceq B^2 f(m) \preceq \frac{B^2}{4}
\end{equation*}

Concluding that $\ell(\boldsymbol{w}, (\boldsymbol{x}, y))$ is $\frac{B^2}{4}$ smooth.
