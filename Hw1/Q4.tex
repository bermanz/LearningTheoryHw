Let $h^*$ be the hypothesis for which $err_D(h^*) > \epsilon$. Cosidering a sample S of size m drawn from D, we define the event
$E = \{\exists x_i \in S^m S.T h_1*(x_i) \neq h_2(x_i)\}$. Given the occurrence of E, we can classify the erroneous hypothesis as $h^*$ and the correct one as $f$. To guarantee this occurrence with probability of at least $1-\delta$, we can upperbound the probability of the complement event by $\delta$:

\begin{equation*}
    \begin{split}
        P(\bar{E}) &= P(\{\forall x_i \in S^m S.T h_1*(x_i) = h_2(x_i)\}) = P(\cup_{i=1}^m h^*(x_i) = f(x_i)) = \prod_{i=1}^m P(h^*(x_i) = f(x_i)) \leq \prod_{i=1}^m (1-\epsilon) = (1-\epsilon)^m \leq e^{-\epsilon m} \overset{!}{<} \delta \\
        m &> \frac{1}{\epsilon} ln(\frac{1}{\delta})
    \end{split}
\end{equation*}

Thus, sampling a set of size m > $\frac{1}{\epsilon} ln(\frac{1}{\delta})$ from D and looking for an erroneous classification guarantees a successful classification w.p. at least $1-\delta$.