\subsubsection{}
\begin{proof}    
    True. let $f_1, f_2, f_3$ be the maximizers of set $\mathcal{F}$ at the arbitrary points $x, y and z=(1-\lambda)x + \lambda y$ respectively. Then:
    \begin{equation*}
        F(z) = f_3(z) \leq (1-\lambda)f_3(x) + \lambda f_3(y) \leq (1-\lambda)f_1(x) + \lambda f_2(y) = (1-\lambda)F(x) + \lambda F(y)
    \end{equation*}
    Where the 1st inequality follows since $f_3$ is a convex function and the 2nd inequality since $f_1, f_2$ are the maximizers of $\mathcal{F}$ at x,y respectively.
\end{proof}

\subsubsection{}
\begin{proof}    
    True. let $h = f + g$, then:
    \begin{equation*}
        \begin{split}
            h((1-\lambda)x + \lambda y) &= f((1-\lambda)x + \lambda y) + g((1-\lambda)x + \lambda y) \\
            &\leq (1-\lambda)f(x) + \lambda f(y) + (1-\lambda)g(x) + \lambda g(y) \\
            &= (1-\lambda)(f(x) + g(x)) + \lambda (f(y) + g(y)) \\
            &= (1-\lambda)h(x) + \lambda h(y)             
        \end{split}
    \end{equation*}
\end{proof}

\subsubsection{}
\begin{proof}    
    True. Let $h = \alpha \cdot f$. Since $f$ is convex, by definition:
    \begin{equation*}
        f((1-\lambda)x + \lambda y) \leq (1-\lambda)f(x) + \lambda f(y)
    \end{equation*}
    Multiplying both sides by $\alpha \geq 0$ won't change the direction of the inequality, hence:
    \begin{equation*}
        \alpha \cdot f((1-\lambda)x + \lambda y) = h((1-\lambda)x + \lambda y) \leq (1-\lambda) \alpha f(x) + \lambda \alpha f(y) = \leq (1-\lambda) h(x) + \lambda h(y)
    \end{equation*}    
\end{proof}

\subsubsection{}
\begin{proof}    
    False. Let $f(z) = z^2, \; g(z) = z^2-1$ and let $h = f \circ g$. Now let $x=1, y=-1, \lambda=\frac{1}{2}$, then:
    \begin{equation*}
        \begin{split}
            h((1-\lambda)x + \lambda y) &= h(0) = (0^2-1)^2 = 1 \\
            (1-\lambda)h(x) + \lambda h(y) &= \frac{1}{2}(h(1) + h(-1)) = \frac{1}{2}((1^2-1)^2 + ((-1)^2-1)^2) = \frac{1}{2} (0 + 0) = 0
        \end{split}
    \end{equation*}
    So we have $h((1-\lambda)x + \lambda y) > (1-\lambda)h(x) + \lambda h(y)$ meaning $h = f \circ g$ isn't convex.
\end{proof}

\subsubsection{}
\begin{proof}
    False. Let $f(z) = z^2, \; g(z) = z^2-1$ and let $h = f \circ g$. Now let $x=1, y=-1, \lambda=\frac{1}{2}$, then:
    \begin{equation*}
        \begin{split}
            h((1-\lambda)x + \lambda y) &= h(0) = (0^2-1)^2 = 1 \\
            (1-\lambda)h(x) + \lambda h(y) &= \frac{1}{2}(h(1) + h(-1)) = \frac{1}{2}((1^2-1)^2 + ((-1)^2-1)^2) = \frac{1}{2} (0 + 0) = 0
        \end{split}
    \end{equation*}
    So we have $h((1-\lambda)x + \lambda y) > (1-\lambda)h(x) + \lambda h(y)$ meaning $h = f \circ g$ isn't convex.
\end{proof}

\subsubsection{}
\begin{proof}
    True. We saw in class that if $f$ is differentiable and convex, the bound:
    \begin{equation*}
        f(y) - f(x) \geq \nabla f(x) \cdot (y-x)
    \end{equation*}
    holds $\forall x,y$ in the domain over which $f$ is defined. W.l.o.g let's assume $x$ be the point in the domain for which $\nabla f(x) = 0$, then:
    \begin{equation*}
        f(y) - f(x) \geq 0 \Rightarrow f(y) \geq f(x), \; \forall y
    \end{equation*}
    meaning that $f(x) = \underset{x'}{min} f(x'), \; \forall x'$.
\end{proof}

\subsubsection{}
\begin{proof}
    False. Let $f(x) = arctan(x)$. Since $f(x)$ is monotonically increasing, the set $A_\alpha$ is always a half closed interval on the real axis $\forall \alpha < \frac{\pi}{2}$, and is the entire real axis for $\alpha \geq \frac{\pi}{2}$, which in all cases is a convex set. Hence, it satisfies the conditions of the question. However, this is clearly not a convex function, since it's strictly concave $\forall x > 0$.
\end{proof}

\subsubsection{}
\begin{proof}
    True. let $h_1,h_2 \in K + G$ then $\exists u_1,u_2 \in K, \; v_1,v_2 \in G$ s.t $h_1=u_1+v_1, \; h_2=u_2+v_2$. Now:
    \begin{equation*}
        \begin{split}            
            \lambda h_1 + (1-\lambda) h_2 &= \lambda (u_1 + v_1) + (1-\lambda) (u_2 + v_2) = 
            \left(\lambda u_1 + (1-\lambda) u_2 \right) +  \left(\lambda v_1 + (1-\lambda) v_2 \right) \\
            &= \tilde{u} + \tilde{v} \in K+G
        \end{split}
    \end{equation*} 
    Where the last transition is since $\tilde{u} \in K$ and $\tilde{v} \in K$ because both are convex combinations of members of the convex sets.
\end{proof}