First, let's show that there exists a set of d+1 points that is shuttered by H. 
In fact, for any set of d+1 different scalars $\{x_i\}_{i=1}^d$, the matrix:
\begin{equation*}
    X = \begin{pmatrix}
        1, x_1, x_1^2, \dotsc  x_1^d \\
        1, x_2, x_2^2, \dotsc  x_2^d \\
        \vdots \\
        1, x_n, x_n^2, \dotsc  x_n^d \\
    \end{pmatrix}
\end{equation*}
is non-singular. hence, for any arbitrary vector $\boldsymbol{y} \in \mathbb{R}^{d+1}$, the linear system:
\begin{equation*}
    X \cdot \boldsymbol{\alpha} = \boldsymbol{y}
\end{equation*}
is solvable and has a unique solution $\boldsymbol{\alpha} = X^{-1}\boldsymbol{y}$. This means that for every arbitrary set of labels $sgn(\boldsymbol{y})$, we can find a set of coefficients $\boldsymbol{\alpha}$, such that:
\begin{equation*}
    sgn(X \cdot \boldsymbol{\alpha}) = sgn(\boldsymbol{y})
\end{equation*}
Which is precisely the definition of a set being shuttered by a concept class.

To show that no set of size $d+2$ can be shuttered by $H$, we'll follow the fundamental theorem of algebra, which states that a polynomial of degree d has exactly d complex roots. Assuming all those roots are real and different, the discussed polynomial has d zero-crossings (in every other case, it has less), partitioning the x-axis to at most $d+1$ classification intervals. Therefore, following the pigeonhole theorem, at least 2 points will will belong in the same interval, and so will have the same label, meaning they can't be arbitrarily classified, and thus aren't shuttered by H.
