Following a similar logic to ex1, We go over the positive examples from left to right. We set $l_1$ to be the first positive example, and $u_1$ the last consecutive positive example (before encountering a negative one). we continue until we cover all examples.
To analyse the error and sample complexity, we first note that if $\sum_{i=1}^k I_{[l_i, u_i]} < \epsilon$, the all negative hypothesis already achieves the desired error rate w.p 1. Otherwise, some of the positive intervals $I_{[l_i, u_i]}$ are of size greater than $\frac{\epsilon}{2k}$, in which case we define the event:
$E^+_i = \{\exists x^+_l,x^+_u \sim D S.T x^+_l \in [l_i, l_i + \frac{\epsilon}{4k}] \cap  x^+_u \in [u_i - \frac{\epsilon}{4k}, u_i]\}$
In addition, forall negative intervals $I^-_i$ greater than $\frac{\epsilon}{4k}$, we define the event:
$E^-_i = \{\exists x^- \sim D S.T x_- \in I^-_i\}$.
Assuming the event $E = E^+_i \cap E^-_i$ occurs, the obtained error is:
\begin{equation*}
    \begin{split}        
        P(err(h_S) | E) &\leq \sum_{i=1}^{k+1} D(I^-_i < \frac{\epsilon}{4k}) + \sum_{i=1}^k D([l_i, l_i + \frac{\epsilon}{4k}]) + D([u_i - \frac{\epsilon}{4k}, u_i]) \\
        &= (k+1) \frac{\epsilon}{4k} + 2k\frac{\epsilon}{4k} \leq 2k\frac{\epsilon}{4k} + 2k\frac{\epsilon}{4k} = 4k\frac{\epsilon}{4k} = \epsilon
    \end{split}
\end{equation*}
As requested. To find the sample complexity, let's upperbound the probability of failure:
\begin{equation*}
    \begin{split}
        P(err(h_S) > \epsilon) &= P(err(h_S) | \bar{E}) \\
        &\leq P(\cup_{i=1}^{k+1} \bar{E}^-_i, \bar{E}^+_i) \sum_{i=1}^{k+1}\leq P(\bar{E}^-_i + \bar{E}^+_i) \leq (k+1)(1-\frac{\epsilon}{4k})^m  + k(1-\frac{\epsilon}{4k})^m \\
        &\leq 4k(1-\frac{\epsilon}{4k})^m \leq 4ke^{-m\frac{epsilon}{4k}} \overset{!}{<} \delta \\ 
        m(\epsilon, \delta) &> \frac{4k}{\epsilon} ln(\frac{4k}{\delta})
    \end{split}
\end{equation*}