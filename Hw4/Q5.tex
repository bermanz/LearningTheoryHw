\subsection*{a}
\subsubsection*{1}
\begin{proof}    
    If we plug in $\boldsymbol{w}=\boldsymbol{0}$, we get:
    \begin{equation*}
        \mathcal{L}_{\lambda, S}(\boldsymbol{w=0}) = \frac{\lambda}{2}||\boldsymbol{0}|| 
        + \frac{1}{m} \sum_{i=1}^{m} max\{0, 1-0\} = \frac{1}{m} \sum_{i=1}^{m} 1 = 1
    \end{equation*}
    
    Now let $\boldsymbol{w}_{\lambda, S} = \underset{\boldsymbol{w}}{argmin} \mathcal{L}_{\lambda, S}(\boldsymbol{w})$, so by construction, $\mathcal{L}_{\lambda, S}(\boldsymbol{w}_{\lambda, S}) \leq \mathcal{L}_{\lambda, S}(\boldsymbol{0}) = 1$, proving the desired inequality
\end{proof}

\subsubsection*{2}
\begin{proof}
    Following subsection (1), we have:
    \begin{equation*}
        \begin{split}            
            & \mathcal{L}_{\lambda, S}(\boldsymbol{w}_{\lambda, S}) = \frac{\lambda}{2}||\boldsymbol{w}_{\lambda, S}|| 
            + \frac{1}{m} \sum_{i=1}^{m} max\{0, 1-y \boldsymbol{w}_{\lambda, S} \cdot \boldsymbol{x}\} \leq 1 \\
            \Rightarrow & \frac{\lambda}{2}||\boldsymbol{w}_{\lambda, S}|| \leq 1 - \frac{1}{m} \sum_{i=1}^{m} max\{0, 1-y \boldsymbol{w}_{\lambda, S} \cdot \boldsymbol{x}\} \leq 1
        \end{split}
    \end{equation*}
    Where the last transition is since $max\{0, 1-y \boldsymbol{w}_{\lambda, S} \cdot \boldsymbol{x}\} \geq 0 \; \forall y, \boldsymbol{w}_{\lambda, S}, \boldsymbol{x}$. The desired result is obtained by multiplying both sides by $\frac{2}{\lambda}$ (which doesn't alter the inequality's direction since $\lambda > 0$) and taking the square root (which also doesn't change the inequality's direction since it's a monotonically increasing function).
\end{proof}

\subsection*{b}
\begin{proof}
    There are 2 possibilities:
    \begin{enumerate}
        \item $sign(\boldsymbol{w}\cdot \boldsymbol{x}) = y$, in which case $\ell_{0, 1}(sign(\boldsymbol{w}\cdot \boldsymbol{x})) = 0$ and $\ell_{\text{hinge}}(\boldsymbol{w}; \boldsymbol{x}, y) = max\{0, 1-\alpha\}$ where $\alpha > 0$. Hence, $\ell_{\text{hinge}}(\boldsymbol{w}; \boldsymbol{x}, y)$ can only take values greater then 0, and so is an upper-bound for $\ell_{0, 1}(sign(\boldsymbol{w}\cdot \boldsymbol{x}), y) = 0$ in this case.
        \item $sign(\boldsymbol{w}\cdot \boldsymbol{x}) \neq y$, in which case $\ell_{0, 1}(sign(\boldsymbol{w}\cdot \boldsymbol{x})) = 1$ and $\ell_{\text{hinge}}(\boldsymbol{w}; \boldsymbol{x}, y) = max\{0, 1 + \alpha\} = 1 + \alpha$ where $\alpha > 0$. Hence, $\ell_{\text{hinge}}(\boldsymbol{w}; \boldsymbol{x}, y)$ can only take values greater then 1, and so is an upper-bound for $\ell_{0, 1}(sign(\boldsymbol{w}\cdot \boldsymbol{x}), y) = 1$ in this case.
    \end{enumerate}
    Concluding that $\ell_{0, 1}(sign(\boldsymbol{w}\cdot \boldsymbol{x}), y) \leq \ell_{\text{hinge}}(\boldsymbol{w}; \boldsymbol{x}, y) \; \forall \boldsymbol{w}$. Clearly, since this holds $\forall y, \boldsymbol{x}$, this is also true in expectation, that is:
    \begin{equation*}
        \mathbb{E} \ell_{0, 1}(sign(\boldsymbol{w}\cdot \boldsymbol{x}), y) \leq \mathbb{E} \ell_{\text{hinge}}(\boldsymbol{w}; \boldsymbol{x}, y)
    \end{equation*}

    Simplifying the expectation for the 0-1 loss, we get:
    \begin{equation*}
        \begin{split}            
            \mathbb{E} \ell_{0, 1}(sign(\boldsymbol{w}\cdot \boldsymbol{x}), y)
            &= \ell_{0, 1}(sign(\boldsymbol{w}\cdot \boldsymbol{x}) \neq y) \cdot D(sign(\boldsymbol{w}\cdot \boldsymbol{x}) \neq y) + \ell_{0, 1}(sign(\boldsymbol{w}\cdot \boldsymbol{x}) = y) \cdot D(sign(\boldsymbol{w}\cdot \boldsymbol{x}) = y) \\
            &= 1 \cdot D(sign(\boldsymbol{w}\cdot \boldsymbol{x}) \neq y) + 0 \cdot D(sign(\boldsymbol{w}\cdot \boldsymbol{x}) = y) = D(sign(\boldsymbol{w}\cdot \boldsymbol{x}) \neq y) 
            = err(\boldsymbol{w})
        \end{split}
    \end{equation*}

    Hence:
    \begin{equation*}
        err(\boldsymbol{w}) \leq \mathbb{E} \ell_{\text{hinge}}(\boldsymbol{w}; \boldsymbol{x}, y)
    \end{equation*}
    As required.
\end{proof}

\subsection*{c}
\begin{proof}
    First, we restrict the domain of $\boldsymbol{w}$ to the ball of radius $B=\sqrt{\frac{2}{\lambda}}$. now, let's find the supremum and Liphschitzness of $\ell_{\text{hinge}}(\boldsymbol{w};(\boldsymbol{x}, y))$|:
    \begin{equation*}
        |\ell_{\text{hinge}}(\boldsymbol{w};(\boldsymbol{x}, y))| \leq |1-y\boldsymbol{w}\cdot\boldsymbol{x}| \leq |1 + \boldsymbol{w}\cdot \frac{\boldsymbol{w}}{||\boldsymbol{w}||}| = 1 + ||\boldsymbol{w}|| \leq 1+\sqrt{\frac{2}{\lambda}} \triangleq c
    \end{equation*}
    \begin{equation*}
        |\nabla \ell_{\text{hinge}}(\boldsymbol{w};(\boldsymbol{x}, y))| \leq |\nabla (1-y\boldsymbol{w}\cdot\boldsymbol{x})| = |-y\boldsymbol{x}| \leq ||\boldsymbol{x}|| \leq 1 \triangleq \rho
    \end{equation*}
    Following the proven theorem from section 4, we have:
    \begin{equation*}
        \underset{\boldsymbol{w} \in K}{sup} \mathcal{L}_D(\boldsymbol{w}) - \mathcal{L}_S(\boldsymbol{w}) \leq 2 \sqrt{\frac{2}{\lambda m}} + O\left((1+\sqrt{\frac{2}{\lambda}}) \sqrt{\frac{ln 1/\delta}{m}} \right)
        = O\left(\sqrt{\frac{ln 1/\delta}{\lambda m}} \right)
    \end{equation*}
    Where the last transition is since we can assume $\lambda \leq 1$, and since for typical (small enough) values of $\delta$, $ln 1/\delta >> 1$.

    Now, following the results from subsection 1, we know that:
    \begin{equation*}
        \mathcal{L}_{\lambda, S}(\boldsymbol{w_{\lambda, S}}) \leq \underset{||\boldsymbol{w}|| \leq 1}{min} \mathcal{L}_{\lambda, S}(\boldsymbol{w})
    \end{equation*}
    Since $w_{\lambda, S}$ is the unconstrained minimizer of $\mathcal{L}_{\lambda, S}(\boldsymbol{w})$. Hence:
    \begin{equation*}
        \begin{split}            
            \mathcal{L}_{\lambda, S}(\boldsymbol{w_{\lambda, S}}) &\leq \underset{||\boldsymbol{w}|| \leq 1}{min} \mathcal{L}_{\lambda, S}(\boldsymbol{w})
            = \underset{||\boldsymbol{w}\leq 1}{min} \frac{\lambda}{2} ||\boldsymbol{w}||^2 + \frac{1}{m} \sum_{i=1}^m \ell_{\text{hinge}}(\boldsymbol{w};(\boldsymbol{x}_i, y_i)) \\
            &\leq \frac{\lambda}{2} + \underset{||\boldsymbol{w}|| \leq 1}{min} \frac{1}{m} \sum_{i=1}^m \ell_{\text{hinge}}(\boldsymbol{w};(\boldsymbol{x}_i, y_i))
        \end{split}
    \end{equation*}
    Thence:
    \begin{equation*}
        \begin{split}
            \mathcal{L}_D(\boldsymbol{w_{\lambda, S}}) - \left(\frac{\lambda}{2} + \underset{||\boldsymbol{w}|| \leq 1}{min} \frac{1}{m} \sum_{i=1}^m \ell_{\text{hinge}}(\boldsymbol{w};(\boldsymbol{x}_i, y_i))\right) 
            &\leq \mathcal{L}_D(\boldsymbol{w_{\lambda, S}}) - \mathcal{L}_{\lambda, S}(\boldsymbol{w_{\lambda, S}}) \\
            &= \mathcal{L}_D(\boldsymbol{w_{\lambda, S}}) - \left(1 + \mathcal{L}_S(\boldsymbol{w_{\lambda, S}})\right) \\
            &\leq \underset{\boldsymbol{w} \in K}{sup} \mathcal{L}_D(\boldsymbol{w}) - \mathcal{L}_S(\boldsymbol{w}) - 1 \\
            &\leq O\left(\sqrt{\frac{ln 1/\delta}{\lambda m}} \right) - 1 = O\left(\sqrt{\frac{ln 1/\delta}{\lambda m}} \right)
        \end{split}
    \end{equation*}
    And by adding $\frac{\lambda}{2}$ to both sides, we obtain:
    \begin{equation*}
        \mathcal{L}_D(\boldsymbol{w_{\lambda, S}}) - \underset{||\boldsymbol{w}|| \leq 1}{min} \frac{1}{m} \sum_{i=1}^m \ell_{\text{hinge}}(\boldsymbol{w};(\boldsymbol{x}_i, y_i)) \leq \frac{\lambda}{2} + O\left(\sqrt{\frac{ln 1/\delta}{\lambda m}} \right) = O\left(\sqrt{\frac{ln 1/\delta}{\lambda m}} + \lambda \right)
    \end{equation*}    
\end{proof}

\subsection*{d}
If a such a $||\boldsymbol{w*}||$ exists, then $\ell_{\text{hinge}}(\boldsymbol{w};(\boldsymbol{x}_i, y_i)) = 0 \forall x_i, y_i \in D$, in which case the result from subsection c collapses to:

\begin{equation*}
    \mathcal{L}_D(\boldsymbol{w_{\lambda, S}}) \leq O\left(\sqrt{\frac{ln 1/\delta}{\lambda m}} + \lambda \right)
\end{equation*}

Based on the result from subsection b, this also means that:
\begin{equation*}
    err(\boldsymbol{w_{\lambda, S}}) \leq O\left(\sqrt{\frac{ln 1/\delta}{\lambda m}} + \lambda \right)
\end{equation*}
Now, we damand that the RHS is less or equal $\epsilon$, and solve for m as function of $\epsilon, \delta$ and $\lambda$:

\begin{equation*}
    \sqrt{\frac{ln 1/\delta}{\lambda m}} + \lambda \overset{!}{\leq} \epsilon 
    \Rightarrow \frac{ln 1/\delta}{\lambda m} \leq (\epsilon-\lambda)^2
    \Rightarrow m \geq \frac{ln 1/\delta}{\lambda(\lambda-\epsilon)^2}
\end{equation*}
Where we can set $\lambda$ to any arbitrary value of choice.
